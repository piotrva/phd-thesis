\chapter{Wstęp}
	Pamięci komputerowe dzielą się na nieulotne i ulotne. Pierwsze z nich mogą przechowywać dane bez konieczności zasilania przez długi okres czasu, ale są względnie wolne, biorąc pod uwagę czas zapisu i odczytu. Pamięci takie posiadają zwykle dużą gęstość zapisu, co uzasadnia ich stosowanie do przechowywania danych.
	
	Pamięci ulotne z kolei potrzebują stałego zasilania w celu przechowywania danych, jednak w zamian oferują bardzo szybki zapis i odczyt danych. Gęstość zapisu w takich pamięciach jest jednak mniejsza niż w wypadku nieulotnych.

	W ostatnich latach zaprezentowane zostały nowe typy pamięci \cite{fujisaki2013review,kent2015new}, które łączą zalety obu przedstawionych powyżej typów, oferując czasy zapisu-odczytu porównywalne z pamięciami ulotnymi, jednocześnie umożliwiając przechowywanie danych po odłączeniu zasilania. Obecnie najszerzej znane typy takich pamięci to FRAM, PCRAM, ReRAM oraz MRAM, których to dotyczy niniejsza praca.
    
    Pamięci MRAM, czyli magnetyczne pamięci RAM, mają praktycznie nieograniczoną ilość cykli zapisu-odczytu, a czasy dostępu moga być porównywane z pamięciami DRAM. Są dodatkowo odporne na promieniowanie jonizujące/kosmiczne. Czyni to je szczególnie interesującymi magaznyami danych do zastosowania w sondach kosmicznych, lotnictwie i innych dziedzinach, wymagających wysokiej niezawodności. Omawianym w tej pracy, jednym z kilku typów pamięci MRAM, jest pamięć STT-MRAM, wykorzystująca transfer spinowego momentu siły (Spin Tarnsfer Torque). Pamięci te są stosunkowo efektywne pod względem energetycznym.
    
    Wadą pamięci STT-MRAM jest jednak stosunkowo niska gęstość zapisu. Wynika ona z faktu, że duży tranzystor, mogący zapewnić odpowiednią gęstość prądu w czasie krótkiego impulsu, steruje wielokrotnie mniejszym elementem pamiętającym. Dlatego ważnym krokiem w kierunku optymalizacji pamięci STT-MRAM jest możliwość sterowania za pomocą jednego tranzystora strukturą mogącą przechowywać wiele bitów - pozwoli to na zwiększenie gęstości zapisu, a zatem pojemności i obszaru stosowania tych pamięci.
    
    Obecnie przedstawiono niewiele pomysłów wykonania wielobitowej struktury pamiętającej, ponieważ skupiano wysiłki albo na zaprojektowaniu jednego elementu, który mógłby być stabilnym w wielu stanach, albo na łączeniu elementów poprzez ich wykonanie jeden na drugim. Oba te podejścia stanowią jednak w obecnyczh czasach wyzwanie dla procesu produkcji.
    
    W tej pracy przedstawiono nowe podejście - przeanalizowano możliwość elektrycznego połączenia równoległego i szeregowego pojedynczych elementów pamiętających STT-MRAM a także przeprowadzono szereg eksperymentów. Jako rezultat końcowy osiągnięto wykonaną eksperymentalnie 3-bitową strukturę pamiętającą opartą na połączonych szeregowo standardowych komórkach STT-MRAM. Daje to możliwość produkcji pamięci o większej gęstości zapisu bez znacznej modyfikacji stosowanych obecnie technik produkcyjnych.
    
    Na rozwiązanie to zgłoszono wniosek patentowy oznaczony numerem P.427097, którego głównym autorem jest autor niniejszej pracy.