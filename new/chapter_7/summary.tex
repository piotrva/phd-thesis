\chapter{Summary}\label{summary}

\indent In this Thesis the realization of low-loss microwave circuits in strip transmission line technique was studied. As a result, the Author proposed several novel design methodologies, circuit topologies and realization schemes. The conducted theoretical studies were supported with experimental results on an example of numerous manufactured circuits. Within the scope of the Thesis, various approaches were considered and investigated with the focus on power loss reduction.
\\
\indent Circuit topologies and realization schemes being an alternative to the  classic solutions, were studied on an example of microwave filters to allow for obtaining a given functionality while featuring improved properties such as highly-selective frequency response, compact physical size or relatively low insertion loss. The periodic structures approach was considered and proven to be well suitable for the realization of filters for which a cascade connection of $n$ identical, electrically small unit cells of given properties is used.
\\
\indent Following, the Author investigated the influence of circuit topology and realization technique on the total power loss on an example of broadband pseudo-highpass microwave filters. It was shown that by replacement of relatively lossy lumped components with distributed type ones as well as by substitution of stripline/microstrip dielectric stack-up to suspended stripline/microstrip stack-up, a significant reduction of power loss can be achieved.
\\
\indent Moreover, in this Thesis design methodologies and circuit topologies focused on the performance improvement were studied to allow for power loss reduction on an example of directional filters and directional couplers. It was shown, that high-performance circuits can meet the design requirements while featuring shorter total electrical length or requiring less physical area, therefore, reducing total power loss.
\\
\indent Furthermore, the Author investigated the approach for power loss reduction within the system by the integration of functionalities within one circuit to reduce the number of components and by increasing level of functional blocks' integration. An example of division/summation networks in terms of power level and frequency spectrum, circuits with simultaneous power division and impedance transformation were shown for the application in compact multi-channel power amplifiers. Moreover, directional-filters-based frequency multiplexers have been proposed, where a high selectivity is obtained by the integration of both filter`s response and multiplexer`s topology properties.
\\
\indent Additionally, the Author considered the application of novel materials and manufacturing technologies for low-loss circuits` realization within microwave frequency range. It was shown that by enabling the 3\textsuperscript{rd} degree of structure's control, the design of compact high-performance circuits is possible as it was shown on an example of directional couplers. Moreover, aspects such as materials` conductivity, dielectric properties and manufacturing technologies have been investigated in terms of requirements for low-loss circuits` realization.
\\
\indent Finally, it was shown that the reduction of power loss allows to extend the circuit functionality on an example of a broad-range impedance tuner for source and load pull transistor measurements. The proposed tuner was proven to be useful for finding optimal terminating impedances for maximum power termination of high-power transistors by the design of an exemplary transceiver front-end.
\\
\indent The original achievements of the Author presented in this Thesis can be summarized as follows:

\begin{itemize}[nosep]
%\renewcommand{\labelitemi}{$\bullet$}
\item Development of compact unit cells composed of coupled and uncoupled transmission line sections allowing for bandpass filters` realization, published in \cite{tmtt_right_left}.
\item Development of a compact unit cell composed of transmission line sections allowing for bandstop filters' realization, published in \cite{mwcl_dcrlh}.
\item Development of pseudo-highpass filters composed of transmission line sections and lumped capacitor unit cells, published in \cite{jemwa_pseudo-highpass}.
\item Development of loss-reduced pseudo-highpass filters composed of transmission line sections and lumped capacitor unit cells, published in \cite{mms_low-loss_wideband}.
\item Development of loss-reduced pseudo-highpass filters composed of unit cells constructed of coupled and uncoupled transmission line sections, published in \cite{mikon_low-loss_distributed}.
\item Development of miniaturized traveling wave directional filters, published in \cite{isap_miniaturized_DF}.
\item Development of traveling wave directional filters with additional, easily controllable transmission zeroes created by the appropriate cascading two identical filters, published in \cite{mikon_cascaded_DF}.
\item Development of low-loss directional filters composed of two differential bandstop filters featuring improved isolation bandwidth, published in \cite{mwcl_band_reject}.
\item Development of traveling-wave directional filters with additional, easily controllable transmission zeroes created by the introduction of loose cross coupling, published in \cite{tmtt_crosscoupled_DF}.
\item Development of the design approach for frequency multiplexers where an increased selectivity of each channel is obtained by the appropriate design of asymmetric directional filters and taking advantage of multiplexer`s topology, published in \cite{mwcl_cascaded_multipex}.
\item Development of an impedance transforming directional coupler with a high impedance transformation ratio, published in \cite{jmwt_imp_tranforming}.
\item Development of a broadband directional coupler composed of loosely coupled single-section directional couplers in tandem configuration, published in \cite{mwcl_tandem}.
%\item Development of an impedance transforming balance-to-unbalance signal converting circuit, published in \cite{jiee_imp_tranf_balun}.
\item Theoretical investigation, confirmed with experimental results on the application of FDM type 3D printing with conductive filaments for the realization of low-loss microwave circuits in suspended microstrip technique, published in \cite{iceese_3D_graphene} and \cite{ectc_electrify}.
\item Development of the realization technique of multilayers microstrip circutis using a combination of additive manufacturing technique, published in \cite{ectc_df}.
\item Development of design methodology with enabled 3\textsuperscript{rd} degree of structure control using 3D printing for the realization of high-performance directional couplers in suspended stripline, published in \cite{polyjet_susp_coupler}.
\item Development of a low-cost impedance tuner with realizable ratio of VWSR as high as 41:1 for load and source pull transistor measurement, published in \cite{rws_imp_tuner}.
\item Development of a high-power RF front-end for ADS-B vehicle transponder and MODE-S interrogator with 20 W peak output power amplifier, described in \cite{avionix_report}.
\end{itemize}

\indent In the view of the above listed achievements being a result of investigatig the presented approaches for power loss reduction it can be concluded, that the goals stated in \hyperref[intro:goal]{Introduction} have been achieved. Various design techniques, circuit topologies and realization schemes have been investigated proving that filters (goal I) and power division/summation circuits such as directional couplers and frequency multiplexers (goal II) can be realized in strip transmission line technique featuring low power loss and high performance. Finally, an exemplary high-power RF front-end was developed (goal III).
\\
\indent The Author believes that the research results presented in the Thesis focused on power loss minimization of circuits realized in strip transmission line technique will contribute to the development of microwave theory and techniques and will be useful for further development of highly integrated functional blocks of wireless telecommunication systems. The replacement of currently used combination of all-metal waveguides and PCB technology with one uniform realization technique will allow for more efficient utilization of the system`s physical volume and better integration of particular building blocks leading to fulfillment of the above stated demands for transmitting and receiving capabilities.
\\
\indent Further research efforts will be focused on the application of 3D printing technologies and novel materials for the realization of low-loss strip transmission line circuits with enabled third dimension of circuit control opening new directions of circuits` and systems` development. The preliminary results presented e.g. in \cite{iceese_3D_graphene}, \cite{ectc_electrify}, \cite{ectc_df}, \cite{polyjet_susp_coupler}, \cite{apwc_absorber}, \cite{iceaa_magnetic} have shown the potential of such an approach. However, there are challenges that needs to be addressed in order to increase the applicability of the 3D printing technology.
\\
\indent Another promising direction of further research can be found in the design and manufacturing of very high frequency circuits where waveguide technique is replaced with strip transmission line technique. It has been shown in \cite{tmtt_aerosol} and \cite{eumw_aerosol} that the application of aerosol jet 3D printing technology in combination with conductive inks and very low loss dielectric inks has a potential for the realization of mm-wave circuits. However, further development is required in terms of materials` and manufacturing processes` optimization.

\cleardoublepage