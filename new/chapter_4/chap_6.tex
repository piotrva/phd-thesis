\chapter{Integration of functionalities and increasing level of integration as a way of power loss reduction}\label{chap6}

\indent In this Chapter it is presented that the reduction of power loss can be obtained when circuits are designed in a way to integrate different functionalities or to increase level of integration. The subject is studied on an example of directional couplers with simultaneous impedance transformation and highly-integrated frequency multiplexer composed of appropriately cascaded directional filters. The results of the conducted research have been a subject of three journal papers: two published in \textit{International Journal of Microwave and Wireless Technologies} and \textit{International Journal of Information and Electronics Engineering} as well as one submitted for \textit{IEEE Microwave and Wireless Components Letters} which constitute the Chapter.
\\
\indent A novel impedance transforming directional couplers are proposed in which the achievable impedance transformation ratio is increased above the previously reported limit related to coupling of the coupled-line sections. The proposed couplers consist of two coupled-line sections between which uncoupled sections of LH lines are connected. The presented concept has been verified by circuit simulations as well as by measurements of the manufactured 3-dB coupled-line impedance transforming directional coupler operating at the center frequency $f\textsubscript{0}$ = 1.2 GHz and featuring twofold increased impedance transformation ratio $R$ = 4.
\\
\indent Following, a new approach to the design of a balun circuit is presented. The proposed balun is composed of an impedance-transforming directional coupler and an LH/RH transmission line differential phase shifter. The proposed solution features broadband amplitude and phase response, moreover, a flexibility of the impedance transformation from balanced to imbalanced port is obtained. The proposed concept has been verified by the design of a balun circuit featuring over-an-octave bandwidth and impedance transformation 70 $\Omega \textsubscript{bal}$ / 50 $\Omega \textsubscript{imbal}$.
\\
\indent Finally, a novel approach to the design of frequency multiplexers consisting of cascaded directional filters is proposed. It is shown that when channels are spaced closely enough, their selectivity can be improved without increasing filters' order by taking advantage of two phenomena. Asymmetric frequency response of a constitutive directional filter allows to increase the attenuation slope on one side of the multiplexer channel. Additionally, the slope on the other side can be increased due to the creation of additional transmission zero resulting from the fact that bandstop response of previous directional filters within the cascade affects directly the response of the following channel. Theoretical analysis is provided together with the applicability condition and multiplexer design procedure. Moreover, an exemplary four-channel S-band multiplexer was manufactured and measured showing the selectivity improvement of as much as $\approx$ 1.3 times.

\cleardoublepage

\includepdf[pages=-,addtotoc={1,section,1,Impedance transforming directional couplers
with increased achievable transformation ratio,tmtt_right_left}]{chapter_6/jmwt.pdf}

\cleardoublepage

\includepdf[pages=-,addtotoc={1,section,1,Broadband balun circuits
composed of impedance transforming directional couplers and LH transmission-line sections,tmtt_right_left}]{chapter_6/jiee.pdf}

\cleardoublepage

\includepdf[pages=-,addtotoc={1,section,1,Frequency multiplexers with
improved selectivity using asymmetric response directional filters,tmtt_right_left}]{chapter_6/mwcl.pdf}


\cleardoublepage