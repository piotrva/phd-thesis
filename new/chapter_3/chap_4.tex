\chapter{Circut power loss reduction by improving subcircuits topology and altering realisation techniqe}\label{chap4}

\indent In this Chapter it is shown that total power loss can be reduced by optimizing circuit`s topology and/or altering the realization technique. The approach is investigated on an example of periodic structure type broadband filters. Results of the conducted research have been a subject of one journal paper published in \textit{Internatonal Journal of Electromagnetic Waves and Applications} and two conference papers presented at \textit{Mediterranean Microwave Conference MMS`17} and submitted for \textit{International Conference on Microwave, Radar and Wireless Communications MIKON`18}, both under auspice of the Institute of Electrical and Electronics Engineers IEEE, which constitute the Chapter.
\\
\indent The design of high selectivity, pseudo-highpass filters is presented. The proposed circuits utilize a novel semi-distributed-element composite right-left handed unit cell composed of sections of transmission lines and a lumped capacitor. By proper balancing the structure, a very broad operational band can be obtained. Moreover, a single-layer microstrip realization is possible making the unit cell well suitable for low-cost filter realization. The proposed concept has been experimentally verified by the design and measurements of an exemplary pseudo-highpass filter.
\\
\indent Following, the realization of low-loss wideband bandpass filters utilizing a periodic structure approach is presented. The recently developed novel semi-distributed composite right-left handed unit cell is considered and adopted for the design of the proposed filters. It is shown that an appropriate balancing of the structure, i.e., the selection of circuit`s parameters allows for achieving very broad passband. Moreover, a suspended microstrip technique is utilized to reduce total insertion loss of the circuit. The presented approach has been confirmed by realization of an exemplary low-loss wideband bandpass filter. The measured operational band is within 1.0 – 9.4 GHz with total loss ranging between 0.35 to 1.8 dB. The obtained results proved the applicability of the presented approach.
\\
\indent Finally, a novel wideband pseudo-highpass filter is presented. Wideband band-pass unit cells and a periodic structure approach are utilized for realization of the filter. Low-losses and circuit properties` control are obtained by the realization of the unit cell using distributed elements only, i.e., transmission line stubs and a coupled-line section as well as the utilization of suspended stripline technique for circuit realization. Theoretical analysis of the unit cell as well as experimental results have been provided. An exemplary manufactured compact low-loss, four-unit-cell filter features 1 - 9 GHz wide fundamental passband with very sharp lower roll-off and minimal insertion losses of 0.24 dB. The obtained results have confirmed the performance of the proposed approach.

\cleardoublepage

\includepdf[pages=-,addtotoc={1,section,1,Pseudo-highpass filters based on
semi-distributed balanced composite right/left-handed unit cells,tmtt_right_left}]{chapter_4/jewa.pdf}

\cleardoublepage

\includepdf[pages=-,addtotoc={1,section,1,Low-loss wideband bandpass filters using
semi-distributed unit cells,tmtt_right_left}]{chapter_4/mms.pdf}

\cleardoublepage

\includepdf[pages=-,addtotoc={1,section,1,Low-loss pseudo-highpass filters using distributed-element unit cells,tmtt_right_left}]{chapter_4/mikon.pdf}

\cleardoublepage