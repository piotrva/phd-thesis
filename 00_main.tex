\documentclass[pdflatex,en,11pt]{aghdpl}
\usepackage[english]{babel}
\usepackage{polski}
\usepackage[utf8]{inputenc}

\usepackage[backend=bibtex,
style=numeric,
sorting=none,
%bibencoding=ascii,
%style=reading,
giveninits=true
]{biblatex}
\renewbibmacro{in:}{%
    \ifentrytype{article}{}{\printtext{\bibstring{in}\intitlepunct}}}

\addbibresource{bibliografia.bib}

\DeclareNameAlias{sortname}{last-first}
\DeclareNameAlias{default}{last-first}

% dodatkowe pakiety
\usepackage{enumerate}
\usepackage{listings}
\usepackage{float}
\usepackage{siunitx}
\usepackage{hyperref}
\usepackage[acronym,nomain,toc]{glossaries}
\usepackage{graphicx}
\usepackage{enumitem}
\usepackage{multirow,bigdelim}
\usepackage{makecell}
\usepackage{subcaption}
\usepackage{multicol}
\usepackage{lipsum}
\usepackage[table]{xcolor}

% \lstloadlanguages{TeX}

\lstset{
  literate={ą}{{\k{a}}}1
           {ć}{{\'c}}1
           {ę}{{\k{e}}}1
           {ó}{{\'o}}1
           {ń}{{\'n}}1
           {ł}{{\l{}}}1
           {ś}{{\'s}}1
           {ź}{{\'z}}1
           {ż}{{\.z}}1
           {Ą}{{\k{A}}}1
           {Ć}{{\'C}}1
           {Ę}{{\k{E}}}1
           {Ó}{{\'O}}1
           {Ń}{{\'N}}1
           {Ł}{{\L{}}}1
           {Ś}{{\'S}}1
           {Ź}{{\'Z}}1
           {Ż}{{\.Z}}1
}

\usepackage{array}
\newcolumntype{L}[1]{>{\raggedright\let\newline\\\arraybackslash\hspace{0pt}}m{#1}}
\newcolumntype{C}[1]{>{\centering\let\newline\\\arraybackslash\hspace{0pt}}m{#1}}
\newcolumntype{R}[1]{>{\raggedleft\let\newline\\\arraybackslash\hspace{0pt}}m{#1}}

\makeatletter
\let\ps@plain\ps@fancy
\makeatother

%---------------------------------------------------------------------------

\author{Piotr Rzeszut}
\shortauthor{P. Rzeszut}

\course{Elektronika i Telekomunikacja}

\titlePL{Magnetyczne złącza tunelowe z anizotropią prostopadłą do zastosowań w szeregowo-równoległych połączeniach elementarnych komórek pamięci stt-mram}
\titleEN{Magnetic tunnel junctions with perpendicular anisotropy for use in serial and parallel connections of elementary stt-mram cells}

\shorttitlePL{Magnetyczne złącza tunelowe z anizotropią prostopadłą do zastosowań w szeregowo-równoległych połączeniach elementarnych komórek pamięci STT-MRAM} % skrócona wersja tytułu jeśli jest bardzo długi
\shorttitleEN{Magnetic tunnel junctions with perpendicular anisotropy for use in serial and parallel connections...}

\thesistypePL{Praca dyplomowa magisterska}
\thesistypeEN{Master Thesis}

\supervisorPL{dr inż. Witold Skowroński}
\supervisorEN{Witold Skowroński Ph.D}

\date{2018}

\departmentPL{Katedra Elektroniki}
\departmentEN{Department of Electronics}

\facultyPL{Wydział Informatyki, Elektroniki i Telekomunikacji}
\facultyEN{Faculty of Computer Science, Electronics and Telecommunications}

\acknowledgements{}

\setlength{\cftsecnumwidth}{10mm}

\DeclareSIUnit\rpm{rpm}

\definecolor{capping}{rgb}{0.45,0.7,0.29}
\definecolor{ferromagnetic}{rgb}{0.25,0.44,0.79}
\definecolor{barrier}{rgb}{0.65,0.65,0.65}
\definecolor{bottom}{rgb}{0.98,0.76,0}
\definecolor{substrate}{rgb}{0.42,0.64,0.86}

\newcommand\varpm{\mathbin{\vcenter{\hbox{%
  \oalign{\hfil$\scriptstyle+$\hfil\cr
          \noalign{\kern-.3ex}
          $\scriptscriptstyle({-})$\cr}%
}}}}
\newcommand\varmp{\mathbin{\vcenter{\hbox{%
  \oalign{\hfil$\scriptstyle-$\hfil\cr
          \noalign{\kern-.3ex}
          $\scriptscriptstyle({+})$\cr}%
}}}}

%---------------------------------------------------------------------------

\begin{document}

\titlepages

%\textbf{\huge CONFIDENTIAL}

    Contents of this work are the subject of patent application and proceedings. Copying, publishing or sharing of any information presented in the work is strictly prohibited until acceptance of the patent application by the patent office.

\vskip 3cm 
\textbf{\huge POUFNE}

    Zawartość niniejszej pracy jest przedmiotem wniosku i postępowania patentowego. Kopiowanie, publikowanie lub rozpowszechnianie jakichkolwiek informacji przedstawianych w tej pracy jest surowo wzbroniona do momentu przyjęcia wniosku patentowego przez urząd patentowy.
\begin{center}
    \thispagestyle{empty}
    \vspace*{\fill}
    Rodzicom
    \vspace*{\fill}
\end{center}
\tableofcontents

\chapter{Definitions and abbreviated terms}

\section{Definitions}
\begin{itemize}
    \item \textbf{Storage element} -- a single magnetic tunnel junction or other magnetoresistive device capable of being stable in two or more resistance states.
    \item \textbf{Storage cell} -- a single storage element or a group of storage elements driven by a single read-write circuit.
    \item \textbf{Multi-bit (storage) cell} -- a storage cell capable of being stable in more than two states, resulting in the ability to store more than one bit of data.
    \item \textbf{Magnetoresistance ratio} -- a measure of the amplitude of magnetoresistive effects, defined as $(R_{max}-R_{min})/R_{min}$.
    \item \textbf{Critical current} -- current that induces switching of the storage element.


\end{itemize}

\section{Abbreviations}
\begin{itemize}
%    \item    \textbf{ASIC}     --    Application Specific Integrated Circuit
%    \item    \textbf{$\mathbf{I^2C}$}   --    Inter-Integrated Circuit
    \item    \textbf{AF}      --    Antiferromagnet
    \item    \textbf{AP}      --    Antiparallel
	\item    \textbf{CIMS}    --    Current Induced Magnetisation Switching
	\item    \textbf{DI}      --    Deionized
	\item    \textbf{FL}      --    Free Layer
	\item    \textbf{FM}      --    Ferromagnet
	\item    \textbf{GPIB}    --    General Purpose Interface Bus (IEEE-488)
	\item    \textbf{IEC}     --    Interlayer Exchange Coupling
	\item    \textbf{IP}      --    In Plane
	\item    \textbf{MR}      --    Magnetoresistance Ratio
    \item    \textbf{MRAM}    --    Magnetic Random Access Memory
    \item    \textbf{MTJ}     --    Magnetic Tunnel Junction
    \item    \textbf{NM}      --    Non-magnetic
    \item    \textbf{P}       --    Parallel
    \item    \textbf{PC}      --    Personal Computer
    \item    \textbf{PL}      --    Pinned Layer
    \item    \textbf{PMA}     --    Perpendicular Magnetic Anisotropy
    \item    \textbf{PSV}     --    Pseudo Spin Valve
    \item    \textbf{RAM}     --    Random Access Memory
    \item    \textbf{RKKY}    --    Ruderman–Kittel–Kasuya–Yosida (coupling)
    \item    \textbf{RL}      --    Reference Layer
    \item    \textbf{RT}      --    Room Temperature
    \item    \textbf{SAF}     --    Synthetic Anti-ferromagnet
    \item    \textbf{SEM}     --    Scanning Electron Microscopy
    \item    \textbf{STT}     --    Spin Transfer Torque
    \item    \textbf{TMR}     --    Tunnel Magnetoresistance
    \item    \textbf{USB}     --    Universal Serial Bus
\end{itemize}

\chapter{Introduction}
    Computer memories can be primarily classified into two groups: volatile and non-volatile memories. Non-volatile memories are able to store data without power applied for a long period of time, but usually are relatively slow in terms of writing speed and data access time. Such memories are for example Hard Drive Disks (HDD) or Flash memories. They exhibit large data density, what makes them perfect for storage of data without a need of fast access.
    
    Volatile memories, in contrast, need constant power supply to preserve data, but offer very fast read and write operations. The data density is smaller than for non-volatile memories.
    
    In the past few years some new types of memories have been presented \cite{fujisaki2013review,kent2015new}, which connect advantages of standard volatile and non-volatile memories. They offer read-write speeds comparable to fast volatile memories, while still being able to preserve data without power. These types include:
    
    \begin{itemize}[noitemsep,label=\textbullet]
    	\item FRAM -- Ferroelectric Random Access Memory (RAM), which is similar to the classic Dynamic RAM (DRAM) memory, but replaces the dielectric with a ferroelectric material
    	\item PCRAM -- Phase Change RAM, which bases on materials, that can change the phase form amorphous to crystal, what is reflected in the change of resistance of the storage element
    	\item ReRAM -- Redox RAM, which bases on redox reactions in the storage element, what is reflected in the change of resistance of the storage element
    	\item MRAM -- Magnetoresistive RAM, which bases on effects such as Tunnel Magnetoresistance, Spin Transfer Torque and other relate to spin electronics
    \end{itemize}
    
    However, considering current state of the art, all these memories have some disadvantages. FRAM memories have very low data density, PCRAM and ReRAM exhibit very limited endurance and the capability of MRAM chips is limited due to the size of the cell driving circuit \cite{kawahara20082}, which is considerably bigger than a storage element.
    
    MRAM memories, which are subject of interest of this work, have the following advantages: their endurance is theoretically unlimited \cite{kent2015new}, their write and read speeds can be compared to DRAM memory (especially the writing speed is much shorter than for FLASH). In addition they are ionizing-cosmic-radiation resistant, which makes them perfect for use in avionics and space electronics. 
    
    There are several types of MRAM memories, taking into account the method of data writing, for example: field-driven (toggle), spin transfer torque (STT-MRAM), domain-wall motion (DW motion) and voltage-controlled spintronics memories (VoCSM) \cite{yoda2016voltage}. The field-driven MRAMs as well as STT-MRAMs are commercially available nowadays, while other types are still the subject of research. Additionally, field-driven memories turn out to be less power efficient and have lower densities than STT memories.
    
    The limitation of data density, mentioned above, is the result of the current needed to switch the STT-MRAM storage element compared to current density allowed in state-of-the-art transistors. Therefore, the ability to drive multiple storage elements using one transistor, resulting in having a multi-bit memory cell, can lead to considerable improvement in terms of MRAMs capability and, as a result, extend the area of their application.
    
    To date, very few practical implementations of multi-bit MRAM cells have been presented. This is mainly due to the fact, that efforts were made to produce a single storage element capable of being stable in more than two states, or to produce multiple storage elements on the top of each other. Both of these approaches are very challenging for the process of manufacture.
    
    In this thesis a new approach is presented. The analysis of serial and parallel connections of multiple standard STT-MRAM elements was performed, as well as experiments were conducted. The final result of this work includes practical realisation of three-bit storage cell based on STT-MRAM elements.
    
    In Sec. \ref{sec:Principles} the physical phenomenon, MRAMs are based on, is introduced to the reader and the approach presented in the thesis justified. Then, in Sec. \ref{sec:Fabrication}, a complete fabrication process of MRAM cell is described. Later, in Sec. \ref{sec:Experiment}, an experiment is described, and results discussed. As a conclusion, future challenges of the presented solutions are briefly described in Sec. \ref{sec:FutureWork}. The work is summarized in Sec. \ref{sec:Summary}.

\chapter{Principles of operation}
\label{sec:Principles}

This chapter describes fundamental physical phenomena:

\lipsum

\section{Magnetic tunnel junction static behaviour} \label{sec:PrinciplesMTJStatic}
\lipsum

\section{Magnetic tunnel junction dynamic behaviour} \label{sec:PrinciplesMTJDynamic}
\lipsum
\chapter{Fabrication techniques}
\label{sec:Fabrication}

This chapter presents MTJ stacks used for the experiment and describes the complete fabrication process of the sample.

\section{MTJ stack deposition for static devices} \label{sec:FabricationStackStatic}

    Based on previous experiments \cite{skowronski2017understanding}, an MTJ layer structure with PMA was proposed and deposited by Singulus AG using TIMARIS sputtering system in $Ar$ atmosphere on an oxidised $Si$ substrate. The substrate was etched before the deposition using ion etching, in order to clean the surface. The layer structure is presented in Tab. \ref{tab:FabricationLayerStructure}. 
    
    Layers 37-35 form a buffer, which reduces the surface roughness and induces proper crystal growth of other layers \cite{banasik2015magnetic}. Layers 34-8 form a SAF, with a reference layer (10-8) on the top. $Co/Pt$ layers 34-21 form a superlattice with strong PMA (Sec. \ref{sec:PrinciplesAdditionalPerpendicular}) which is coupled antiferromagnetically to another $Co/Pt$ superlattice (19-12) through a \SI{0.8}{\nano\meter} $Ru$ spacer (20). A composite reference layer (10-8) is also coupled through thin W layer (which, in addition, serves as a texture break), completing the SAF. A \SI{0.89}{\nano\meter} thick $MgO$ is used as a tunnel barrier (7), which results in the resistance area (RA) product of around \SI{20}{\ohm\times\micro\metre\squared}. Above, a composite free layer is placed (6-4). An important part of the top capping (3-1) is another $MgO$ layer (3), which increases PMA of the free layer.
    
    After the deposition process the sample was annealed at \SI{380}{\celsius} for \SI{60}{\minute}. The process allowed to relax interface stresses of the layers.
    
    \begin{table}[H]
    	\caption{Layer structure of the sample used for experiment, from top to bottom.}
    	\label{tab:FabricationLayerStructure}

    	\begin{center}
    	  \begin{tabular}{r r l l@{\hspace{20pt}} l}
    	    No. & Material & \multicolumn{3}{l}{Thickness (\SI{}{\nano\meter})} \\ \hline
    	    1 & $Ru$ 	& \cellcolor{capping}5.00 & \rdelim\}{3}{3mm}[Top capping] \\
    	    2 & $Ta$ 	& \cellcolor{capping}3.00 \\
    	    3 & $MgO$ 	& \cellcolor{capping}1.00 \\ \hline
    	    4 & $CoFeB$ & \cellcolor{ferromagnetic}0.50 & \rdelim\}{3}{3mm}[Free layer] \\
    	    5 & $W$ 	& \cellcolor{ferromagnetic}0.30 \\
    	    6 & $CoFeB$ & \cellcolor{ferromagnetic}1.30 \\ \hline
    	    7 & $MgO$	& \cellcolor{barrier}0.89 & Tunnel barrier \\ \hline
    	    8 & $CoFeB$ & \cellcolor{ferromagnetic}1.00 & \rdelim\}{3}{3mm}[Reference layer] & \rdelim\}{13}{3mm}[SAF] \\
    	    9 & $W$		& \cellcolor{ferromagnetic}0.25 \\
    	    10& $Co$	& \cellcolor{ferromagnetic}0.90 \\
    	    11& $Ta$	& \cellcolor{ferromagnetic}0.15 \\
    	    12& $Pt$	& \cellcolor{ferromagnetic}0.20 \\
    	   \ldelim\{{2}{3mm}[13\textasciitilde 18]& $Co$ & \cellcolor{ferromagnetic}0.50 & \rdelim\}{2}{3mm}[$\times 3$] \\
    	    &   $Pt$	& \cellcolor{ferromagnetic}0.20 \\
    	    19& $Co$	& \cellcolor{ferromagnetic}0.60 \\
    	    20& $Ru$	& \cellcolor{ferromagnetic}0.80 \\
    	    21& $Co$	& \cellcolor{ferromagnetic}0.60 \\
    	    \ldelim\{{2}{3mm}[22\textasciitilde 33] & $Pt$ & \cellcolor{ferromagnetic}0.20 & \rdelim\}{2}{3mm}[$\times 6$] \\
    	    &   $Co$	& \cellcolor{ferromagnetic}0.50 \\
    	    34& $Pt$	& \cellcolor{ferromagnetic}1.50 \\ \hline
    	    35& $Ta$	& \cellcolor{bottom}0.70 & \rdelim\}{3}{3mm}[Bottom buffer] \\
    	    36& $Ru$	& \cellcolor{bottom}7.00 \\
    	    37& $Ta$	& \cellcolor{bottom}2.00 \\ \hline
    	    & $SiO_2$ & \cellcolor{substrate} & \rdelim\}{2}{3mm}[Oxidised substrate]\\
    	    & $Si$ & \cellcolor{substrate} \\
    	  \end{tabular}
    	\end{center}
    \end{table}

\section{MTJ stack deposition for dynamic devices} \label{sec:FabricationStackDynamic}
\lipsum

\chapter{Electrical measurements}
\label{sec:Experiment}

    This chapter describes the experimental setup, and presents the results of the measurements taken.
    
\section{Static characterisation} \label{sec:ExperimentStatic}
\lipsum
   
\section{Dynamic characterisation} \label{sec:ExperimentDynamic}
\lipsum
\chapter{Future work}
\label{sec:FutureWork}

    The measurements' results revealed, that the biggest issue with such design of multi-bit storage cell is the narrow region for writing higher resistances, due to the stochastic nature of the switching process of the cells with very similar parameters. A solution to the problem may be to intentionally vary some of the parameters, by using serially connected elements of different sizes. By varying the size, the critical current and resistance of the element may be changed.

    Another vital improvement is to optimise the nanostructurization process in order to minimize the number of non-working elements. This improvement would increase chances to test bigger storage cells as well as produce fully operational miniaturised serial connections.

    With the increased number of storage elements in a storage cell, higher voltage is needed to switch the cell. Another field of improvement would be to decrease resistance of the element, maximize TMR as well as minimise the critical current. Such improvements would lead to the decrease of supply voltage of the memory, but they are most difficult to implement, due to the complex layer structure of the MTJ.

\chapter{Summary}
\label{sec:Summary}

    By analysing the theoretical behaviour of MTJa described in the literature, the performance of serially and parallelly connected elements was predicted. Serial connection was believed to act as a multi-bit storage cell, and was selected for experimental verification.

    The experiment involved fabrication of multi-bit storage cells, utilizing the previously developed MTJ layer structure. Two- and three-bit storage cells were successfully tested, proving that the theoretical analysis was correct.
    
    The developed method of manufacturing and driving multi-bit non-volatile storage elements is a significant improvement in MRAM technology, as it allows to store more data using the same area of the memory. This is achieved by driving a multi-bit storage cell using a single transistor rated for the same current, as a single storage element (the critical current remains the same for any number of serially connected elements). Also, the manufacturing process, utilizing vias fabricated in the same step as the MTJ pillar, does not require significant changes compared to single storage element fabrication.
    
    For now, significant drawbacks of the storage cell are quite narrow regions for switching subsequent elements to the AP state and relatively high voltages involved. The first one may be overcome by using serially connected elements of slightly different sizes. The other drawback needs careful MTJ stack redesign.
    
    In conclusion, the presented work is an important step to develop a functional STT-MRAM memory with increased data density, and proves, that multi-bit storage elements may be easily manufactured. Also, the presented solution is the subject of patent proceedings.
    

\chapter{Acknowledgements}

	This work is supported by the Polish Ministry of Science and Higher Education \textbf{Diamond Grant No. 0048/DIA/2017/46} and the Polish National Center for Research and Development grant No. \textbf{LIDER/467/L-6/14/NCBR/2015}.
	
    \vspace{2cm}
    
    I would like to express my profound gratitude to \textbf{my parents}, for their support and invaluable help at every moment of my life.

    I would like to thank my supervisor, \textbf{PhD Witold Skowroński}, for his help in preparing the work and carrying out the experiments.

	I am very grateful to \textbf{Professor Tomasz Stobiecki}, for giving me an opportunity to explore the world of science.

    I would like to thank \textbf{PhD Sławomir Ziętek}, for his help in carrying out the experiments.
    
    I am grateful to \textbf{Academic Center of Materials and Nanotechnology (ACMiN AGH)} and \textbf{Professor Marek Przybylski} for an opportunity to perform nanostructurization of the sample. I would also like to thank \textbf{Singulus AG} and \textbf{PhD Jerzy Wrona} for providing the sample with MTJ stack deposited.





%\appendix
\printbibliography

\end{document}
