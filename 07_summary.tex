\chapter{Summary}
\label{sec:Summary}

    By analysing the theoretical behaviour of MTJa described in the literature, the performance of serially and parallelly connected elements was predicted. Serial connection was believed to act as a multi-bit storage cell, and was selected for experimental verification.

    The experiment involved fabrication of multi-bit storage cells, utilizing the previously developed MTJ layer structure. Two- and three-bit storage cells were successfully tested, proving that the theoretical analysis was correct.
    
    The developed method of manufacturing and driving multi-bit non-volatile storage elements is a significant improvement in MRAM technology, as it allows to store more data using the same area of the memory. This is achieved by driving a multi-bit storage cell using a single transistor rated for the same current, as a single storage element (the critical current remains the same for any number of serially connected elements). Also, the manufacturing process, utilizing vias fabricated in the same step as the MTJ pillar, does not require significant changes compared to single storage element fabrication.
    
    For now, significant drawbacks of the storage cell are quite narrow regions for switching subsequent elements to the AP state and relatively high voltages involved. The first one may be overcome by using serially connected elements of slightly different sizes. The other drawback needs careful MTJ stack redesign.
    
    In conclusion, the presented work is an important step to develop a functional STT-MRAM memory with increased data density, and proves, that multi-bit storage elements may be easily manufactured. Also, the presented solution is the subject of patent proceedings.
    