\chapter{Fabrication techniques}
\label{sec:Fabrication}

This chapter presents MTJ stacks used for the experiment and describes the complete fabrication process of the sample.

\section{MTJ stack deposition for static devices} \label{sec:FabricationStackStatic}

    Based on previous experiments \cite{skowronski2017understanding}, an MTJ layer structure with PMA was proposed and deposited by Singulus AG using TIMARIS sputtering system in $Ar$ atmosphere on an oxidised $Si$ substrate. The substrate was etched before the deposition using ion etching, in order to clean the surface. The layer structure is presented in Tab. \ref{tab:FabricationLayerStructure}. 
    
    Layers 37-35 form a buffer, which reduces the surface roughness and induces proper crystal growth of other layers \cite{banasik2015magnetic}. Layers 34-8 form a SAF, with a reference layer (10-8) on the top. $Co/Pt$ layers 34-21 form a superlattice with strong PMA (Sec. \ref{sec:PrinciplesAdditionalPerpendicular}) which is coupled antiferromagnetically to another $Co/Pt$ superlattice (19-12) through a \SI{0.8}{\nano\meter} $Ru$ spacer (20). A composite reference layer (10-8) is also coupled through thin W layer (which, in addition, serves as a texture break), completing the SAF. A \SI{0.89}{\nano\meter} thick $MgO$ is used as a tunnel barrier (7), which results in the resistance area (RA) product of around \SI{20}{\ohm\times\micro\metre\squared}. Above, a composite free layer is placed (6-4). An important part of the top capping (3-1) is another $MgO$ layer (3), which increases PMA of the free layer.
    
    After the deposition process the sample was annealed at \SI{380}{\celsius} for \SI{60}{\minute}. The process allowed to relax interface stresses of the layers.
    
    \begin{table}[H]
    	\caption{Layer structure of the sample used for experiment, from top to bottom.}
    	\label{tab:FabricationLayerStructure}

    	\begin{center}
    	  \begin{tabular}{r r l l@{\hspace{20pt}} l}
    	    No. & Material & \multicolumn{3}{l}{Thickness (\SI{}{\nano\meter})} \\ \hline
    	    1 & $Ru$ 	& \cellcolor{capping}5.00 & \rdelim\}{3}{3mm}[Top capping] \\
    	    2 & $Ta$ 	& \cellcolor{capping}3.00 \\
    	    3 & $MgO$ 	& \cellcolor{capping}1.00 \\ \hline
    	    4 & $CoFeB$ & \cellcolor{ferromagnetic}0.50 & \rdelim\}{3}{3mm}[Free layer] \\
    	    5 & $W$ 	& \cellcolor{ferromagnetic}0.30 \\
    	    6 & $CoFeB$ & \cellcolor{ferromagnetic}1.30 \\ \hline
    	    7 & $MgO$	& \cellcolor{barrier}0.89 & Tunnel barrier \\ \hline
    	    8 & $CoFeB$ & \cellcolor{ferromagnetic}1.00 & \rdelim\}{3}{3mm}[Reference layer] & \rdelim\}{13}{3mm}[SAF] \\
    	    9 & $W$		& \cellcolor{ferromagnetic}0.25 \\
    	    10& $Co$	& \cellcolor{ferromagnetic}0.90 \\
    	    11& $Ta$	& \cellcolor{ferromagnetic}0.15 \\
    	    12& $Pt$	& \cellcolor{ferromagnetic}0.20 \\
    	   \ldelim\{{2}{3mm}[13\textasciitilde 18]& $Co$ & \cellcolor{ferromagnetic}0.50 & \rdelim\}{2}{3mm}[$\times 3$] \\
    	    &   $Pt$	& \cellcolor{ferromagnetic}0.20 \\
    	    19& $Co$	& \cellcolor{ferromagnetic}0.60 \\
    	    20& $Ru$	& \cellcolor{ferromagnetic}0.80 \\
    	    21& $Co$	& \cellcolor{ferromagnetic}0.60 \\
    	    \ldelim\{{2}{3mm}[22\textasciitilde 33] & $Pt$ & \cellcolor{ferromagnetic}0.20 & \rdelim\}{2}{3mm}[$\times 6$] \\
    	    &   $Co$	& \cellcolor{ferromagnetic}0.50 \\
    	    34& $Pt$	& \cellcolor{ferromagnetic}1.50 \\ \hline
    	    35& $Ta$	& \cellcolor{bottom}0.70 & \rdelim\}{3}{3mm}[Bottom buffer] \\
    	    36& $Ru$	& \cellcolor{bottom}7.00 \\
    	    37& $Ta$	& \cellcolor{bottom}2.00 \\ \hline
    	    & $SiO_2$ & \cellcolor{substrate} & \rdelim\}{2}{3mm}[Oxidised substrate]\\
    	    & $Si$ & \cellcolor{substrate} \\
    	  \end{tabular}
    	\end{center}
    \end{table}

\section{MTJ stack deposition for dynamic devices} \label{sec:FabricationStackDynamic}
\lipsum
